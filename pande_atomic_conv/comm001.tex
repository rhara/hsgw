\documentclass{jarticle}
\begin{document}
\thispagestyle{empty}

長谷川様、

\vspace{1cm}

Pandeの論文(\texttt{https://arxiv.org/abs/1703.10603})の追試に四苦八苦しています。

\begin{enumerate}
  \item Distance matrix and neighbor list construction
  \vspace{-3mm}
  \item Atom type convolution
  \vspace{-3mm}
  \item Radial pooling layer
  \vspace{-3mm}
  \item Atomistic fully connected network
\end{enumerate}

のうち、1.、2.は簡単で高速な処理をnumpyで実装しました。

\[
E_{i,j,n_a} = (K^a*R)_{i,j}
\]

この式では$E$のshapeは$(N, M, N_{\mathrm{at}})$となりますが、
分子の大きさに依存しないよう、axis $=0$方向にpaddingを加えています。
コード\texttt{pande\_aconv.py}の中では以下の部分になります。

\begin{verbatim}
analytic_ligand = AnalyticMol.FromFile(args.ligand)
ligand_size = 70
E_ligand = analytic_ligand.getConvolution(padding=ligand_size)
\end{verbatim}

【質問】AtomicConvolutionの初期フィーチャーである$E(N, M, N_{\textrm{at}})$を
多くのリガンド:蛋白で蓄積して
データセット$X(-1, N, M, N_{\textrm{at}})$を用意することはできます。
ここからdeepchem/tensorflowに依存しないで、機械学習のシステムは構築できますでしょうか?
\vspace{.5cm}

3.のpoolingのところは論文内の記述も不備で、
ましてやradial cutoff関数$f_c(r_{i,j})$も明らかに間違いです。

\begin{eqnarray}
  f_c(r_{i,j}) = \left\{
      \begin{array}{ll}
      \frac12\left(\cos\left(\frac{\pi r_{i,j}}{R_c}\right) + 1\right) & 0 < r_{i,j} < R_c \\
      0 & r_{i,j} \geq R_c \\
      \end{array} \right.
\end{eqnarray}

shape $(N, N_{\mathrm{at}}, N_r)$のテンソル$P$を計算する手法はわかりましたが、
スケーリングの$\beta_{n_r}$やバイアス$b_{n_r}$の設定がわかりません。
また、radial symmetry関数 (プーリング関数) $f_s(r_{i,j})$中の$r_s$と$\sigma_s$は学習パラメータなのですが、
どのように取り扱うべきなのかまだよくわかっていません。
論文中の以下の表記は明らかに誤解を与える表現です。

\begin{quote}
Parameter $R_c$ is the radial interaction cutoff, which is fixed to 12 \AA.
\end{quote}

実際には、コードではradialの等差数列$[1.5, 2.0, 2.5, ..., 11.5, 12.0]$です。

\end{document}
